\input{../doc-class-cours.tex}

\begin{document}

\textbf{Nom, Prénom :} \hspace{8cm} \textbf{Classe :} \hspace{3cm} \textbf{Date :}\\

\subsection*{DM3 - Appolo.}

À partir d'une recherche à l'aide d'un moteur de recherche. On utilisera principalement Wikipédia.

\url{https://fr.wikipedia.org/wiki/Programme_Apollo#Les_missions_lunaires}


\begin{enumerate}
  \item[1.]Que signifie les lettres N. A. S. et A. dans NASA ? \\
  \Pointilles[2]

  \item[2.]Appolo est le nom du \dotfill
  
  \item[3.]Quelles sont les 7 programmes Appolo qui se sont posés sur la lune ? \textit{Préciser l'année} \\
  \Pointilles[4]

  \item[4.]Qui sont les deux premiers astronautes à avoir marché sur la Lune ? \\
  \Pointilles[1]

  Quelle phrase a rendu célèbre Armstrong lors de ses premiers pas sur la Lune ? \\
  \Pointilles[2]

  Et en anglais. \\
  \Pointilles[2]

  \item[5.]Combien de personnes ont marché sur la Lune ? \\
  \Pointilles[3]
\end{enumerate}

\subsection*{Et la suite ?}

À partir d'une recherche à l'aide d'un moteur de recherche. 

\begin{itemize}
  \item On utilisera les mots clés : \textbf{Prochaine mission lune}
  \item Dès qu'on connaît le nom de la mission, on peut l'utiliser directement dans le moteur de recherche.
\end{itemize}
\url{https://www.lemonde.fr/sciences/article/2022/09/03/mission-artemis-1-deuxieme-essai-de-decollage-vers-la-lune-pour-la-megafusee-de-la-nasa_6140099_1650684.html}

\begin{enumerate}
  \item[a.] Quel est le nom de la prochaine mission sur la Lune ? \dotfill
  \item[b.] Quelle est la date du prochain lancement ? \dotfill
  \item[c.] Cette mission est-elle habitée ? \dotfill
  \item[d.] Combien de temps le voyage dure-t-il ? \dotfill
\end{enumerate}

\newpage

\textbf{Nom, Prénom :} \hspace{8cm} \textbf{Classe :} \hspace{3cm} \textbf{Date :}\\

\subsection*{DM3 - Appolo.}

À partir d'une recherche à l'aide d'un moteur de recherche. On utilisera principalement Wikipédia.

\url{https://fr.wikipedia.org/wiki/Programme_Apollo#Les_missions_lunaires}


\begin{enumerate}
  \item[1.]Que signifie les lettres N. A. S. et A. dans NASA ? \\
  \Pointilles[2]

  \item[2.]Appolo est le nom du \dotfill
  
  \item[3.]Quelles sont les 7 programmes Appolo qui se sont posés sur la lune ? \textit{Préciser l'année} \\
  \Pointilles[4]

  \item[4.]Qui sont les deux premiers astronautes à avoir marché sur la Lune ? \\
  \Pointilles[1]

  Quelle phrase a rendu célèbre Armstrong lors de ses premiers pas sur la Lune ? \\
  \Pointilles[2]

  Et en anglais. \\
  \Pointilles[2]

  \item[5.]Combien de personnes ont marché sur la Lune ? \\
  \Pointilles[3]
\end{enumerate}

\subsection*{Et la suite ?}

À partir d'une recherche à l'aide d'un moteur de recherche. 

\begin{itemize}
  \item On utilisera les mots clés : \textbf{Prochaine mission lune}
  \item Dès qu'on connaît le nom de la mission, on peut l'utiliser directement dans le moteur de recherche.
\end{itemize}
\url{https://www.lemonde.fr/sciences/article/2022/09/03/mission-artemis-1-deuxieme-essai-de-decollage-vers-la-lune-pour-la-megafusee-de-la-nasa_6140099_1650684.html}

\begin{enumerate}
  \item[a.] Quel est le nom de la prochaine mission sur la Lune ? \dotfill
  \item[b.] Quelle est la date du prochain lancement ? \dotfill
  \item[c.] Cette mission est-elle habitée ? \dotfill
  \item[d.] Combien de temps le voyage dure-t-il ? \dotfill
\end{enumerate}

\end{document}
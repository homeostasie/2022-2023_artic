\input{../doc-class-cours.tex}

\begin{document}

\textbf{Nom, Prénom :} \hspace{8cm} \textbf{Classe :} \hspace{3cm} \textbf{Date :}\\

\vspace{-0.5cm} \begin{center}
  \textit{La valeur morale ne peut pas être remplacée par la valeur intelligence et j'ajouterai : Dieu merci !}  - \textbf{Albert Einstein}
\end{center}

\begin{multicols}{2}

\textbf{Pb1 - Bateau}

\begin{figure}[H]
  \centering
  \includegraphics[width=0.6\linewidth]{4x10-trigonometrie/pb1.pdf}
\end{figure}

\begin{enumerate}
  \item Entre la tour et le phare, le bateau avance à la vitesse de $6m/s$ pendant 2min. \\
  Quelle est la distance ?
  \item Pour être en sécurité, le bateau doit être à une distance plus grande que $350m$ de la côté. \\
  Le bateau est-il en sécurité ?
\end{enumerate} \columnbreak

\Pointilles[17]

\end{multicols}

\textbf{Pb2 - Terre}

\begin{multicols}{2}

  \begin{figure}[H]
    \centering
    \includegraphics[width=0.6\linewidth]{4x10-trigonometrie/pb2.pdf}
  \end{figure}
  
  Le Rayon de la Terre est $6 371 km$. 
  \begin{enumerate}
    \item Calculer la distance AF.
    \item Calculer le périmètre du \textit{parallèle} passant par F.
    \item La terre fait un tour sur elle-même en 24h. \\
    Calculer la vitesse à lequel tourne le point F en une journée.
  \end{enumerate} \columnbreak

  \Pointilles[18]

\end{multicols}

\Pointilles[8]


\end{document}

\documentclass[11pt]{article}
\usepackage{geometry,marginnote} % Pour passer au format A4
\geometry{hmargin=1cm, vmargin=1cm} % 

% Page et encodage
\usepackage[T1]{fontenc} % Use 8-bit encoding that has 256 glyphs
\usepackage[english,french]{babel} % Français et anglais
\usepackage[utf8]{inputenc} 

\usepackage{lmodern,numprint}
\setlength\parindent{0pt}

% Graphiques
\usepackage{graphicx,float,grffile,units}
\usepackage{tikz,pst-eucl,pst-plot,pstricks,pst-node,pstricks-add,pst-fun,pgfplots} 

% Maths et divers
\usepackage{amsmath,amsfonts,amssymb,amsthm,verbatim}
\usepackage{multicol,enumitem,url,eurosym,gensymb,tabularx}

\DeclareUnicodeCharacter{20AC}{\euro}



% Sections
\usepackage{sectsty} % Allows customizing section commands
\allsectionsfont{\centering \normalfont\scshape}

% Tête et pied de page
\usepackage{fancyhdr} \pagestyle{fancyplain} \fancyhead{} \fancyfoot{}

\renewcommand{\headrulewidth}{0pt} % Remove header underlines
\renewcommand{\footrulewidth}{0pt} % Remove footer underlines

\newcommand{\horrule}[1]{\rule{\linewidth}{#1}} % Create horizontal rule command with 1 argument of height

\newcommand{\Pointilles}[1][3]{%
  \multido{}{#1}{\makebox[\linewidth]{\dotfill}\\[\parskip]
}}

\newtheorem{Definition}{Définition}

\usepackage{siunitx}
\sisetup{
    detect-all,
    output-decimal-marker={,},
    group-minimum-digits = 3,
    group-separator={~},
    number-unit-separator={~},
    inter-unit-product={~}
}

\setlength{\columnseprule}{1pt}

\begin{document}

\textbf{Nom, Prénom :} \hspace{8cm} \textbf{Classe :} \hspace{3cm} \textbf{Date :}\\

\begin{center}
  \textit{Le plus grand ennemi de la connaissance n'est pas l'ignorance. C'est l'illusion de la connaissance.} 
  
  \textbf{Stephen Hawking}
\end{center}

\textbf{Ex 1 : Représenter et Calculer}

\textit{Écrire la puissance à l'aide de l'opération $\times$ puis calculer.}

\begin{multicols}{2}
  \begin{itemize}
    \item[a.] $5^3 + 4 $ \\
              =  \dotfill \\
              =  \dotfill            
    \item[b.] $4 \times 10^4 $ \\
              =  \dotfill \\
              =  \dotfill      
    \item[c.] $5^3 \times 4^2$ \\
              =  \dotfill \\
              =  \dotfill      
    \item[d.] $(-8)^{4}$ \\
              =  \dotfill \\
              =  \dotfill       
  \end{itemize}

\end{multicols}

\textbf{Ex 2 : Calculer}

  \begin{itemize}
    \item[g.] $\dfrac{0,25 \times 10^{-4} \times 1,7 \times 10^{2}}{500 \times (10^5)^2} =  \dotfill $
  \end{itemize}

\textbf{Ex 3 : Les règles de calculs}

\textit{Écrire le résultat sous la forme d'une puissance en utilisant les règles.}

\begin{multicols}{4}
  \begin{enumerate}
  \item[i.] $6^{7}  \times  6^{8}  =  \dotfill$
  \item[j.] $\dfrac{10^{12}}{10^{10}} = \dotfill$
  \item[k.] $5^{6} \times 5^{5} = \dotfill$
  \item[l.] $\dfrac{11^{11}}{11^{6}} = \dotfill$
  \item[m.] $11^{9} \times 7^{9} = \dotfill$
  \item[n.] $5^{6} \times 4^{6} = \dotfill$
  \item[o.] $(12^{10})^{8} = \dotfill$
  \item[p.] $(10^{10})^{7} = \dotfill$
  \end{enumerate}
\end{multicols}

\textbf{Ex 4 : Démontrer}

\textit{Écrire la démonstration pour $(2^4)^3 = 2^{12}$} \\
\Pointilles[5]

\textbf{Ex 5 : Écrire sous forme scientifique.}

\begin{multicols}{2}
  \begin{enumerate}
  \item[q.] $\SI{400000}{} = \dotfill$
  \item[r.] $\SI{342000000}{} = \dotfill$
  \item[s.] $\SI{-23400000}{} = \dotfill$
  \item[t.] $\SI{-14500000}{} = \dotfill$
  \end{enumerate}
\end{multicols}


\textbf{Problème 1 - Masse d'un trou noir}

La masse de la terre est $M_T = 5,9 \times 10^{24} kg$. La masse d'un trou noir est $\SI{30000000}{}$ fois plus lourde. 

\textbf{Quelle est la masse d'un trou noir ?}

\Pointilles[4]

\textbf{Problème 2 - Vitesse de la lumière}

La lumière se déplace à la vitesse de $3 \times 10^8$ m/s. 

\textbf{Quelle distance parcourt-elle en 10 jours ?}

\Pointilles[4]

\newpage

\textbf{Problème 3 - Étoiles Vs Grain de sables.}

Il y a $2^{30}$ galaxies dans notre univers. Chaque galaxies contient $3^{31}$ étoiles.  \\
On estime le volume de sable sur Terre à $1\,200 \text{ milliards de } m^3$. Chaque $m^3$ contient environ $510 \text{milliards}$ de grains de sable. 

\textbf{John affirme qu'il y a plus de grains de sable sur Terre que d'étoiles dans l'univers ? A-t-il raison ?}

\Pointilles[5]

\textbf{Problème 4 - Casa de Papel}

\textit{\og El Professeur \fg{} } vient de dérober 26 millions d’euros. \\
Les billets de banque ont une épaisseur de $60 \times 10^{-6} m$. (On dit 60 micromètres)

\textbf{Quelle hauteur atteindrait une pile de billets de banque de 20 \euro{} représentant cette somme ?}

\Pointilles[6]

\textbf{Problème 5 - $CO_2$}

Yasmine fait une expérience de Chimie. Une molécule de dioxyde de carbone est composée d'un atome de carbone et de deux atomes d'oxygène : $CO_2$. La masse d'un atome de carbone est $ m_c = 2 \times 10^{-26}kg$ et la masse d'un atome d'oxygène est $ m_O = 1.8 \times 10^{-26}kg$. 

\textbf{Combien trouve-t-on de molécules de dioxyde carbone dans 4 kg ?}

\Pointilles[6]

\textbf{Problème 6 - La légende de l'échiquier}

\og On place un grain de riz sur la première case d'un échiquier. Si on fait en sorte de doubler à chaque case le nombre de grains de la case précédente : un grain sur la première case, deux sur la deuxième, quatre sur la troisième, etc.,

\begin{itemize}
    \item[1.] Sachant qu'il y a 64 cases, combien de grains de riz obtient-on sur la dernière case ? 
    \item[2.] On estime que le nombre de grain de riz total sur l'échiquier est : $T = 2^{64} - 1$. \\
    Faire ce calcul. La soustraction est-elle effectuée par la calculatrice ?
    \item[3.] La masse d'un grain de riz est $0.05 \times 10^{-3} kg$. Quelle est la masse totale de l'échiquier ? 
\end{itemize}

\Pointilles[10]

\newpage


\textbf{Nom, Prénom :} \hspace{8cm} \textbf{Classe :} \hspace{3cm} \textbf{Date :}\\

\begin{center}
  \textit{Le plus grand ennemi de la connaissance n'est pas l'ignorance. C'est l'illusion de la connaissance.} 
  
  \textbf{Stephen Hawking}
\end{center}

\textbf{Ex 1 : Représenter et Calculer}

\textit{Écrire la puissance à l'aide de l'opération $\times$ puis calculer.}

\begin{multicols}{2}
  \begin{itemize}
    \item[a.] $3^4 + 1 $ \\
              =  \dotfill \\
              =  \dotfill            
    \item[b.] $6 \times 10^6 $ \\
              =  \dotfill \\
              =  \dotfill      
    \item[c.] $7^2 \times 4^3$ \\
              =  \dotfill \\
              =  \dotfill      
    \item[d.] $(-4)^{4}$ \\
              =  \dotfill \\
              =  \dotfill       
  \end{itemize}

\end{multicols}

\textbf{Ex 2 : Calculer}

  \begin{itemize}
    \item[g.] $\dfrac{0,35 \times 10^{-4} \times 6,7 \times 10^{2}}{500 \times (10^5)^2} =  \dotfill $
  \end{itemize}

\textbf{Ex 3 : Les règles de calculs}

\textit{Écrire le résultat sous la forme d'une puissance en utilisant les règles.}

\begin{multicols}{4}
  \begin{enumerate}
  \item[i.] $8^{4}  \times  8^{10}  =  \dotfill$
  \item[j.] $\dfrac{10^{12}}{10^{2}} = \dotfill$
  \item[k.] $15^{4} \times 15^{11} = \dotfill$
  \item[l.] $\dfrac{12^{14}}{12^{6}} = \dotfill$
  \item[m.] $2^{9} \times 7^{9} = \dotfill$
  \item[n.] $5^{6} \times 4^{6} = \dotfill$
  \item[o.] $(99^{10})^{8} = \dotfill$
  \item[p.] $(14^{10})^{7} = \dotfill$
  \end{enumerate}
\end{multicols}

\textbf{Ex 4 : Démontrer}

\textit{Écrire la démonstration pour $(5^4)^2 = 5^8$} \\
\Pointilles[5]

\textbf{Ex 5 : Écrire sous forme scientifique.}

\begin{multicols}{2}
  \begin{enumerate}
  \item[q.] $\SI{30000}{} = \dotfill$
  \item[r.] $\SI{442000000}{} = \dotfill$
  \item[s.] $\SI{-1340000}{} = \dotfill$
  \item[t.] $\SI{-245000}{} = \dotfill$
  \end{enumerate}
\end{multicols}


\textbf{Problème 1 - Masse d'un trou noir}

La masse de la terre est $M_T = 5,9 \times 10^{24} kg$. La masse d'un trou noir est $\SI{20000000}{}$ fois plus lourde. 

\textbf{Quelle est la masse d'un trou noir ?}

\Pointilles[4]

\textbf{Problème 2 - Vitesse de la lumière}

La lumière se déplace à la vitesse de $3 \times 10^8$ m/s. 

\textbf{Quelle distance parcourt-elle en sept jours ?}

\Pointilles[4]

\newpage

\textbf{Problème 3 - Étoiles Vs Grain de sables.}

Il y a $2^{30}$ galaxies dans notre univers. Chaque galaxies contient $3^{31}$ étoiles.  \\
On estime le volume de sable sur Terre à $1\,100 \text{ milliards de } m^3$. Chaque $m^3$ contient environ $510 \text{milliards}$ de grains de sable. 

\textbf{John affirme qu'il y a plus de grains de sable sur Terre que d'étoiles dans l'univers ? A-t-il raison ?}

\Pointilles[5]

\textbf{Problème 4 - Casa de Papel}

\textit{\og El Professeur \fg{} } vient de dérober 22 millions d’euros. \\
Les billets de banque ont une épaisseur de $60 \times 10^{-6} m$. (On dit 60 micromètres)

\textbf{Quelle hauteur atteindrait une pile de billets de banque de 50 \euro{} représentant cette somme ?}

\Pointilles[6]

\textbf{Problème 5 - $CO_2$}

Yasmine fait une expérience de Chimie. Une molécule de dioxyde de carbone est composée d'un atome de carbone et de deux atomes d'oxygène : $CO_2$. La masse d'un atome de carbone est $ m_c = 2 \times 10^{-26}kg$ et la masse d'un atome d'oxygène est $ m_O = 1.8 \times 10^{-26}kg$. 

\textbf{Combien trouve-t-on de molécules de dioxyde carbone dans 2 kg ?}

\Pointilles[6]

\textbf{Problème 6 - La légende de l'échiquier}

\og On place un grain de riz sur la première case d'un échiquier. Si on fait en sorte de doubler à chaque case le nombre de grains de la case précédente : un grain sur la première case, deux sur la deuxième, quatre sur la troisième, etc.,

\begin{itemize}
    \item[1.] Sachant qu'il y a 64 cases, combien de grains de riz obtient-on sur la dernière case ? 
    \item[2.] On estime que le nombre de grain de riz total sur l'échiquier est : $T = 2^{64} - 1$. \\
    Faire ce calcul. La soustraction est-elle effectuée par la calculatrice ?
    \item[3.] La masse d'un grain de riz est $0.05 \times 10^{-3} kg$. Quelle est la masse totale de l'échiquier ? 
\end{itemize}

\Pointilles[10]

\end{document}

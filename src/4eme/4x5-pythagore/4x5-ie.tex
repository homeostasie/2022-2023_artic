\input{../doc-class-cours.tex}

\begin{document}

\textbf{Nom, Prénom :} \hspace{8cm} \textbf{Classe :} \hspace{3cm} \textbf{Date :}\\

\vspace{-0.8cm}

\begin{center}
  \textit{Si nous faisions tout ce dont nous sommes capables, nous nous surprendrions vraiment.}  - \textbf{Thomas Edison}
\end{center}

\vspace{-0.8cm}

\begin{minipage}[t]{0.65\textwidth}
\subsubsection*{Cours}
\textbf{Théorème de Pythagore : } \dotfill \\
\Pointilles[1]

\end{minipage}
\begin{minipage}[t]{0.35\textwidth}
\subsubsection*{Calculer}

$\sqrt{\dfrac{\sqrt{35}}{\sqrt{13}} \times \sqrt{128} + 150} = \dotfill$
\end{minipage}


\subsubsection*{Pythagore rapide}
\textbf{Écrire le calcul et le résultat.}
  
\begin{figure}[H]
  \centering
  \includegraphics[width=0.6\linewidth]{4x5-pythagore/ex2.png}
\end{figure}

\begin{multicols}{5}
  \begin{enumerate}
  \item[a.] \dotfill \\ \Pointilles[1]
  \item[b.] \dotfill \\ \Pointilles[1]
  \item[c.] \dotfill \\ \Pointilles[1]
  \item[d.] \dotfill \\ \Pointilles[1]
  \item[e.] \dotfill \\ \Pointilles[1]
  \end{enumerate}
\end{multicols}


\subsubsection*{Pythagore Rédaction}

\begin{multicols}{2}
\begin{enumerate}
  \item[a.]Soit RTL un triangle rectangle en R tel que : RT = 37,5 cm et TL = 48,5 cm. \\
  \textbf{Calculer la longueur RL.}

  \item[b.]Soit LFD un triangle rectangle en D tel que : LF = 54,8 m et DL = 42 m. \\
  \textbf{Calculer la longueur FD.}

\end{enumerate}
\end{multicols}

\Pointilles[13]

\begin{minipage}[t]{0.65\textwidth}
  \textbf{pb1.} Mario souhaite sauver la princesse Peach en haut d'un château haut de 45m. Pour cela, il jette un grappin par dessus les douves qui ont une longueur de 35m. Calculer la longueur nécessaire du grappin.
  
  \Pointilles[5]
  \end{minipage}
  \begin{minipage}[t]{0.35\textwidth}
  \begin{figure}[H]
    \centering
    \includegraphics[width=0.5\linewidth]{4x5-pythagore/pb1a.pdf}
  \end{figure}
\end{minipage}

\newpage

\textit{but princess is in another castle...}

\textbf{pb2.} Mario doit grimper à l'échelle pour aller sauver la princesse. Pour être en sécurité une échelle de 8m doit être écarté de 2,5m du mur. À quelle hauteur maximale peut se trouver la princesse ? \\
\Pointilles[5]

\begin{minipage}[t]{0.65\textwidth}
  \textit{but princess is in another castle...}

  \textbf{pb3.}  \textit{À Pise vers 1200 après J. C. (problème attribué à Léonard de Pise, dit Fibonacci, mathématicien italien   du moyen âge).} \\
  Pour battre Bowser, Mario doit utiliser une lance. Une lance de 4m est posée verticalement le long d’une tour considérée comme perpendiculaire au sol. On éloigne l’extrémité de la lance qui repose sur le sol de 2m de la tour. Combien descend l’autre extrémité de la lance le long du mur ?
  \Pointilles[6]
  \end{minipage}
  \begin{minipage}[t]{0.35\textwidth}
  \begin{figure}[H]
    \centering
    \includegraphics[width=0.7\linewidth]{4x5-pythagore/pb3a.pdf}
  \end{figure}
\end{minipage}

\Pointilles[2]

\textit{but princess is in another castle...}

\textbf{pb4.}  Mario doit traverser un pont mais celui-ci n'est pas assez solide. Il doit fixer 10 renforts en bois sur les poteaux verticaux qui soutiennent le ponton, comme le montre le dessin. Calculer la longueur d'un renfort puis calculer la longueur totale de tous les renforts nécessaires. 
  
\begin{figure}[H]
  \centering
  \includegraphics[width=0.6\linewidth]{4x5-pythagore/pb4a.png}
\end{figure}
\Pointilles[10]

\newpage


\textbf{Nom, Prénom :} \hspace{8cm} \textbf{Classe :} \hspace{3cm} \textbf{Date :}\\

\vspace{-0.8cm}

\begin{center}
  \textit{Si nous faisions tout ce dont nous sommes capables, nous nous surprendrions vraiment.}  - \textbf{Thomas Edison}
\end{center}

\vspace{-0.8cm}

\begin{minipage}[t]{0.65\textwidth}
\subsubsection*{Cours}
\textbf{Théorème de Pythagore : } \dotfill \\
\Pointilles[1]

\end{minipage}
\begin{minipage}[t]{0.35\textwidth}
\subsubsection*{Calculer}

$\sqrt{\dfrac{\sqrt{65}}{\sqrt{42}} \times \sqrt{89} + 331} = \dotfill$
\end{minipage}


\subsubsection*{Pythagore rapide}
\textbf{Écrire le calcul et le résultat.}
  
\begin{figure}[H]
  \centering
  \includegraphics[width=0.6\linewidth]{4x5-pythagore/ex2.png}
\end{figure}

\begin{multicols}{5}
  \begin{enumerate}
  \item[a.] \dotfill \\ \Pointilles[1]
  \item[b.] \dotfill \\ \Pointilles[1]
  \item[c.] \dotfill \\ \Pointilles[1]
  \item[d.] \dotfill \\ \Pointilles[1]
  \item[e.] \dotfill \\ \Pointilles[1]
  \end{enumerate}
\end{multicols}

\subsubsection*{Pythagore Rédaction}

\begin{multicols}{2}
\begin{enumerate}
  \item[a.]Soit XML un triangle rectangle en X tel que : XL = 27,5 cm et XM = 28,5 cm. \\
  \textbf{Calculer la longueur ML.}

  \item[b.]Soit NFT un triangle rectangle en F tel que : NT = 14,8 m et NF = 12 m. \\
  \textbf{Calculer la longueur FT.}

\end{enumerate}
\end{multicols}

\Pointilles[13]

\begin{minipage}[t]{0.65\textwidth}
  \textbf{pb1.} Mario souhaite sauver la princesse Peach en haut d'un château haut de 50m. Pour cela, il jette un grappin par dessus les douves qui ont une longueur de 45m. Calculer la longueur nécessaire du grappin.
  
  \Pointilles[5]
  \end{minipage}
  \begin{minipage}[t]{0.35\textwidth}
  \begin{figure}[H]
    \centering
    \includegraphics[width=0.5\linewidth]{4x5-pythagore/pb1b.pdf}
  \end{figure}
\end{minipage}

\newpage

\textit{but princess is in another castle...}

\textbf{pb2.} Mario doit grimper à l'échelle pour aller sauver la princesse. Pour être en sécurité une échelle de 7m doit être écarté de 2,2m du mur. À quelle hauteur maximale peut se trouver la princesse ? \\
\Pointilles[5]

\begin{minipage}[t]{0.65\textwidth}
  \textit{but princess is in another castle...}

  \textbf{pb3.}  \textit{À Pise vers 1200 après J. C. (problème attribué à Léonard de Pise, dit Fibonacci, mathématicien italien   du moyen âge).} \\
  Pour battre Bowser, Mario doit utiliser une lance. Une lance de 8m est posée verticalement le long d’une tour considérée comme perpendiculaire au sol. On éloigne l’extrémité de la lance qui repose sur le sol de 4m de la tour. Combien descend l’autre extrémité de la lance le long du mur ?
  \Pointilles[6]
  \end{minipage}
  \begin{minipage}[t]{0.35\textwidth}
  \begin{figure}[H]
    \centering
    \includegraphics[width=0.7\linewidth]{4x5-pythagore/pb3b.pdf}
  \end{figure}
\end{minipage}

\Pointilles[2]

\textit{but princess is in another castle...}

\textbf{pb4.}  Mario doit traverser un pont mais celui-ci n'est pas assez solide. Il doit fixer 22 renforts en bois sur les poteaux verticaux qui soutiennent le ponton, comme le montre le dessin. Calculer la longueur d'un renfort puis calculer la longueur totale de tous les renforts nécessaires. 
  
\begin{figure}[H]
  \centering
  \includegraphics[width=0.6\linewidth]{4x5-pythagore/pb4b.png}
\end{figure}
\Pointilles[10]




\end{document}

\documentclass[11pt]{article}
\usepackage{geometry,marginnote} % Pour passer au format A4
\geometry{hmargin=1cm, vmargin=1cm} % 

% Page et encodage
\usepackage[T1]{fontenc} % Use 8-bit encoding that has 256 glyphs
\usepackage[english,french]{babel} % Français et anglais
\usepackage[utf8]{inputenc} 

\usepackage{lmodern,numprint}
\setlength\parindent{0pt}

% Graphiques
\usepackage{graphicx,float,grffile,units}
\usepackage{tikz,pst-eucl,pst-plot,pstricks,pst-node,pstricks-add,pst-fun,pgfplots} 

% Maths et divers
\usepackage{amsmath,amsfonts,amssymb,amsthm,verbatim}
\usepackage{multicol,enumitem,url,eurosym,gensymb,tabularx}

\DeclareUnicodeCharacter{20AC}{\euro}



% Sections
\usepackage{sectsty} % Allows customizing section commands
\allsectionsfont{\centering \normalfont\scshape}

% Tête et pied de page
\usepackage{fancyhdr} \pagestyle{fancyplain} \fancyhead{} \fancyfoot{}

\renewcommand{\headrulewidth}{0pt} % Remove header underlines
\renewcommand{\footrulewidth}{0pt} % Remove footer underlines

\newcommand{\horrule}[1]{\rule{\linewidth}{#1}} % Create horizontal rule command with 1 argument of height

\newcommand{\Pointilles}[1][3]{%
  \multido{}{#1}{\makebox[\linewidth]{\dotfill}\\[\parskip]
}}

\newtheorem{Definition}{Définition}

\usepackage{siunitx}
\sisetup{
    detect-all,
    output-decimal-marker={,},
    group-minimum-digits = 3,
    group-separator={~},
    number-unit-separator={~},
    inter-unit-product={~}
}

\setlength{\columnseprule}{1pt}

\begin{document}

\textbf{Nom, Prénom :} \hspace{8cm} \textbf{Classe :} \hspace{3cm} \textbf{Date :}\\

\vspace{-0.8cm}

\begin{center}
  \textit{L’expérience : c’est là le fondement de toutes nos connaissances.}  - \textbf{John Locke}
\end{center}

\vspace{-0.8cm}
\section*{DM Probabilités - Les nombres aléatoires}

\subsection*{1- Aléatoire}

\begin{enumerate}
    \item[1a.] Traduction anglaise du mot aléatoire : \dotfill
    \item[1b.] Abréviation du mot RAM dans l'expression informatique Mémoire RAM : \dotfill
    \item[1c] Quelle groupe français de musique a publié un album portant ce nom : \dotfill
    \item[1d.] À partir d'une recherche sur Internet, qu’est-ce que la mémoire RAM dans un ordinateur ? \\ \Pointilles[8]
\end{enumerate} 


\subsection*{2 - Génération de nombres aléatoires}

\begin{enumerate}
    \item[2a.] À partir d'une recherche sur Internet, qu'est-ce que Mersenne Twister ? \\ \Pointilles[8]
    \item[2b.] Définition : Un nombre premier est \dotfill \\ \Pointilles[1]
    \item[2c.] On nous dit que la période de Mersenne Twister est le nombre premier : $2^{19937} - 1$.  Réécrire ce calcul à l'aide de l'opération multiplier (\textit{rappel chapitre Puissances}) et en utilisant des pointillés. \\

    $2^{19937} - 1 = $ \dotfill
    \item[2d.] Votre calculatrice ne peut pas faire ce calcul. Pourquoi ? \\ \Pointilles[2]
\end{enumerate} 

\newpage

\subsection*{3 - Algorithmes}

Mersenne Twister est un algorithme. 

\begin{enumerate}
    \item[1a.] À partir d'une recherche sur Internet, Qu’est-ce qu'un algorithme ? \\
    (\textit{Remarque : la page \url{https://www.cnil.fr/fr/definition/algorithme} me semble pertinente.}) \\
    \Pointilles[8]

    \item[1b.] À partir d'une recherche sur Internet sur \url{http://lwh.free.fr/pages/algo/crypto/prng.html}, quel est le nom des quatre générateurs de nombres pseudo-aléatoires ?
    
    \begin{itemize}[label={$\bullet$}]
        \item \dotfill
        \item \dotfill
        \item \dotfill
        \item \dotfill
    \end{itemize} 

    \item[1c.] À partir d'une recherche sur Internet, qui est Leonardo Fibonacci et pourquoi est-il connu ? \\ \Pointilles[8]
\end{enumerate} 

\subsection*{4 - Cryptographie}

Plusieurs fois le mot cryptographie est apparu. 

\begin{enumerate}
    \item[4a.] À partir d'une recherche sur Internet, qu'est-ce que la cryptographie ? \\

(\textit{Remarque : la page suivante me semble pertinente :} \\
\url{https://www.cnil.fr/fr/comprendre-les-grands-principes-de-la-cryptologie-et-du-chiffrement} \\ \Pointilles[6]
\end{enumerate} 
\end{document}

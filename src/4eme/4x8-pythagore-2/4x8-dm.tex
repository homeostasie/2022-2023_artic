\input{../doc-class-cours.tex}

\begin{document}

\textbf{Nom, Prénom :} \hspace{8cm} \textbf{Classe :} \hspace{3cm} \textbf{Date :}\\

\vspace{-0.5cm} \begin{center}
  \textit{L'intelligence est insipide sans altruisme.}  - \textbf{Michel Bouthot}
\end{center}

\textbf{Théorème de Pythagore : } \dotfill \\
\Pointilles[1]

\subsubsection*{Ex1 : Pythagore rapide}
\textbf{Écrire le calcul et le résultat.}
  
\begin{figure}[H]
  \centering
  \includegraphics[width=0.6\linewidth]{4x8-pythagore-2/ex2.png}
\end{figure}

  \begin{enumerate}
  \item[a.] \dotfill 
  \item[b.] \dotfill 
  \item[c.] \dotfill 
  \item[d.] \dotfill 
  \item[e.] \dotfill 
  \end{enumerate}


\begin{multicols}{2}

  \subsubsection*{Ex2 : Pythagore Rédaction}

  Soit RTL un triangle rectangle en R tel que : RT = 37,5 cm et TL = 48,5 cm. \\
  \textbf{Calculer la longueur RL.}

  \Pointilles[16] \columnbreak

  \subsubsection*{Ex3 : Pythagore Réciproque}

  Soit RTL un triangle tel que : RT = 15,5 cm, RL = 25,5cm et TL = 20,5 cm. \\
  \textbf{Le triangle est-il rectangle ?}

\Pointilles[16]

\end{multicols}

\newpage

\subsubsection*{pb1. Second poteau Pavard} 

\begin{multicols}{2}

Lors du match de foot France - Argentine du 30 Juin 2018, Pavard (P) met une sublime reprise de volée sur une passe décisive de Lucas (L). La balle terminant dans le but adverse (B).

\url{https://www.youtube.com/watch?v=9nECmaDC5A4}

Pour analyse de l'action, on récupère des données. 

\begin{itemize}
  \item Distance entre Pavard et Lucas : PL = 45m
  \item Distance entre Pavard et le But : PB = 22m
  \item Distance entre le But et Lucas : BL = 40m
\end{itemize}

\textbf{Le triangle formé par Pavard, Lucas et le But est-il rectangle ?} \columnbreak

\begin{figure}[H]
  \centering
  \includegraphics[width=\linewidth]{4x8-pythagore-2/pb1.pdf}
\end{figure}

\end{multicols}

\Pointilles[10]

\begin{multicols}{2}

\subsubsection*{pb2. The Wind Waker} 

Dans The Wind Waker, Link doit réparer la voile triangulaire de son bateau. Pour cela, il doit coudre un fils le long des trois côtés du triangle. 

\textbf{Calculer le périmètre du triangle formé par la voile.} 

\Pointilles[9] \columnbreak

\begin{figure}[H]
  \centering
  \includegraphics[width=0.8\linewidth]{4x8-pythagore-2/pb2.pdf}
\end{figure}

\end{multicols}

\Pointilles[7]

\end{document}
\input{../doc-class-cours.tex}

\begin{document}

\textbf{Nom, Prénom :} \hspace{8cm} \textbf{Classe :} \hspace{3cm} \textbf{Date :}\\
\vspace{-0.8cm}
\begin{center}
  \textit{La normalité est une route pavée : on y marche aisément mais les fleurs n’y poussent pas.} - \textbf{Vincent Van Gogh}
\end{center}
\vspace{-0.8cm}

\subsection*{Démonstrations}

\begin{multicols}{2}
\begin{enumerate}
  \item[1.] Démontrer le résultat de $\dfrac{2}{17} + \dfrac{3}{17}$.  \\ \Pointilles[6] \columnbreak 
  \item[2.] Démontrer le résultat de $\dfrac{2}{17} \times 3$.  \\ \Pointilles[6]
\end{enumerate} 
\end{multicols}

\subsection*{Calculer}

\begin{itemize}[label={$\bullet$}]
\item $\dfrac{6}{25} + \dfrac{10}{25} =$ \dotfill \\
\item $\dfrac{12}{32} + \dfrac{14}{32} - \dfrac{5}{32} =$ \dotfill \\
\item $\dfrac{4}{22} \times 5 + \dfrac{11}{22} =$ \dotfill \\
\item $\dfrac{7}{3} + \dfrac{3}{7} =$ \dotfill \\
\item $\dfrac{25}{12} - \dfrac{25}{12} + 20 =$ \dotfill \\
\item $4 \times \dfrac{6}{11} + \dfrac{16}{11} =$ \dotfill \\
\end{itemize} 

\subsection*{Définition}
La définition mathématique de la fraction $\dfrac{125}{254}$ est : \dotfill \\ \Pointilles[2]

\begin{multicols}{2}
\subsection*{Valeur exacte ou valeur approchée ?}
\begin{itemize}[label={$\bullet$}]
  \item $\dfrac{4}{5}$ \dotfill \\
  \item $\dfrac{120}{3}$ \dotfill \\
  \item $\dfrac{2}{3}$ \dotfill \\
  \item $\dfrac{84}{100}$ \dotfill \\
  \item $\dfrac{73}{21}$ \dotfill \\
  \item $\dfrac{137}{44}$ \dotfill \\
\end{itemize}  \columnbreak 


\subsection*{Sudoku}
\begin{figure}[H]
  \centering
  \includegraphics[width=0.8\linewidth]{6x5-fractions/sudoku-6.png}
\end{figure}

\end{multicols}
\end{document}
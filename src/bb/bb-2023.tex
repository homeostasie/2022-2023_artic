\documentclass[11pt]{article}
\usepackage{geometry,marginnote} % Pour passer au format A4
\geometry{hmargin=1cm, vmargin=1cm} % 

% Page et encodage
\usepackage[T1]{fontenc} % Use 8-bit encoding that has 256 glyphs
\usepackage[english,french]{babel} % Français et anglais
\usepackage[utf8]{inputenc} 

\usepackage{lmodern,numprint}
\setlength\parindent{0pt}

% Graphiques
\usepackage{graphicx,float,grffile,units}
\usepackage{tikz,pst-eucl,pst-tree,pst-plot,pstricks,pst-node,pstricks-add,pst-fun,pst-text,pst-all,pgfplots} 
\usepackage{pgfplots}
% Maths et divers
\usepackage{amsmath,amsfonts,amssymb,amsthm,verbatim}
\usepackage{multicol,enumitem,url,eurosym,gensymb,tabularx}

\DeclareUnicodeCharacter{20AC}{\euro}

\pgfplotsset{compat=newest}

% Sections
\usepackage{sectsty} % Allows customizing section commands
\allsectionsfont{\centering \normalfont\scshape}

% Tête et pied de page
\usepackage{fancyhdr} \pagestyle{fancyplain} \fancyhead{} \fancyfoot{}

\renewcommand{\headrulewidth}{0pt} % Remove header underlines
\renewcommand{\footrulewidth}{0pt} % Remove footer underlines

\newcommand{\horrule}[1]{\rule{\linewidth}{#1}} % Create horizontal rule command with 1 argument of height

\newcommand{\Pointilles}[1][3]{%
  \multido{}{#1}{\makebox[\linewidth]{\dotfill}\\[\parskip]
}}

\newtheorem{Definition}{Définition}

\usepackage{siunitx}
\sisetup{
    detect-all,
    output-decimal-marker={,},
    group-minimum-digits = 3,
    group-separator={~},
    number-unit-separator={~},
    inter-unit-product={~}
}

\setlength{\columnseprule}{1pt}


\begin{document}

\begin{titlepage}

\center % Center everything on the page

\textsc{\LARGE Collège Faubert}\\[2cm] % Name of your university/college
%\textsc{\Large }\\[0.5cm] % Major heading such as course name
\textsc{\large Villefranche}\\[2cm] % Minor heading such as course title

\horrule{2px}

\vspace{1cm}

{ \Huge \bfseries Brevet blanc}\\[2cm] % Title of your document
{ \Huge \bfseries Mathématiques}\\[2cm] % Title of your document
{\large \bfseries 2023}\\[2cm] 

\horrule{2px}

\vspace{1cm}

\begin{itemize}[label={$\bullet$}]
  \item \textsc{Exercice 1} - 13 points     
  \item \textsc{Exercice 2} - 15 points 
  \item \textsc{Exercice 3} - 10 points 
  \item \textsc{Exercice 4} - 13 points 
  \item \textsc{Exercice 5} - 14 points 
  \item \textsc{Exercice 6} - 19 points 
  \item \textsc{Exercice 7} - 16 points 
\end{itemize}

\vspace{1cm}

\horrule{2px}

\vspace{1cm}

\textbf{L'annexe est à rendre avec sa copie. Écrire son numéro de candidat.}

\vspace{1cm}

\begin{itemize}
  \item L'usage de la calculatrice de type collège est autorisé.
  \item L'usage de tout autre document est interdit. 
\end{itemize}

\vfill 

\end{titlepage}


\subsection*{Exercice 1 - 13 points}

La marée est un mouvement régulier et périodique des eaux des océans. Elle est due à l'attraction exercée par la lune et le soleil sur ces eaux. Le niveau le plus haut atteint par la mer au cours d'un cycle de marée est appelé \textit{marée haute} ou \textit{pleine mer}. Le niveau le plus bas se nomme \textit{marée basse} ou \textit{basse mer}.

Le graphique ci-dessous représente un cycle de marée au Mont-Saint-Michel.

\begin{center}
\begin{tikzpicture}
    \begin{axis}%
      [grid=both,
       minor tick num=3,
       grid style={line width=.1pt, draw=gray!30},
       major grid style={line width=.2pt,draw=gray!90},
       axis lines=middle,
       xmin=5,xmax=17,ymin=0,ymax=10,
       xtick={5,6,7,8,9,10,11,12,13,14,15,16,17},
       xticklabels={5,6,7,8,9,10,11,12,13,14,15,16,17},
       ytick={0,1,2,3,4,5,6,7,8,9,10},
       yticklabels={0,1,2,3,4,5,6,7,8,9,10},
       xlabel=Heures de la journée,
       ylabel=Hauteur d'eau en mètre
      ]
      \addplot[domain=5:17,samples=50,smooth,red] {2*cos(deg(pi*(x+9)/6)) + 7};
    \end{axis}
  \end{tikzpicture}
\end{center}

\begin{enumerate}
    \item Quelle est la hauteur d'eau en mètre à 13 heures ?
    \item À quelles heures de la journée la hauteur de l'eau était-elle de 7.5 mètres ?
    \item Quelle était la hauteur d'eau en mètre à 7h45 ?
    \item À quelle heure a eu lieu la marée basse ?
    \item Au cours d'un cycle de marée, la différence de hauteur entre la pleine mer et la basse mer s'appelle le \textit{marnage}. \\
    Déterminer le marnage ce jour là au Mont-Saint-Michel.
\end{enumerate}

\subsection*{Exercice 2 - 15 points}


\medskip

\emph{Pour illustrer l'exercice, la figure ci-dessous a été faite à main levée.}

\begin{center}\psset{unit=1cm}
\begin{pspicture}(8,5)
    \pslineByHand(0.5,2.5)(7.5,4)
    \pslineByHand(2,0.5)(6,4.8)
    \pslineByHand(2,0.5)(0.5,2.5)
    \pslineByHand(6,4.8)(7.5,4)
    \pslineByHand(3,3)(4,2.6)
    \uput[ul](4.75,3.4){A}\uput[r](7.5,4){B}\uput[u](6,4.8){C}
    \uput[l](0.5,2.5){D} \uput[d](2,0.5){E}\uput[ul](3,3){F}\uput[dr](3.8,2.6){G}
    \rput{-50}(1.2,1.2){8,1 cm}\rput{-27}(3.3,2.6){3 cm}
    \rput{46}(3.2,1.4){6,8 cm}\rput{45}(4.4,2.78){4 cm}
    \rput{15}(3.6,3.4){5 cm}\rput{45}(5.2,4.3){5 cm}
    \rput{15}(6,3.4){6,25 cm}
\end{pspicture}
\end{center}

Les points D, F, A et B sont alignés, ainsi que les points E, G, A et C.

De plus, les droites (DE) et (FG) sont parallèles.

\medskip

\begin{enumerate}
\item Montrer que le triangle AFG est un triangle rectangle.
\item Calculer la longueur du segment [AD]. En déduire la longueur du segment [FD].
\item Les droites (FG) et (BC) sont-elles parallèles ? Justifier.
\end{enumerate}

\newpage

\subsection*{Exercice 3 - 10 points}

\medskip

Les représentations graphiques $C_1$ et $C_2$ de deux fonctions sont données dans le repère
ci-dessous.

Une de ces deux fonctions est la fonction $f$ définie par $f(x) = -2x + 8$.

\medskip

\parbox{0.40\linewidth}{\psset{unit=0.6cm}
\begin{pspicture*}(-2,-1.5)(8.5,12.5)
    \psgrid[gridlabels=0,subgriddiv=1,gridwidth=0.2pt]
    \psaxes[linewidth=1.25pt,labelFontSize=\scriptstyle]{->}(0,0)(-2,-1.5)(8.5,12.5)
    \psplot[plotpoints=2000,linewidth=1.25pt]{-2}{4.5}{8 2 x mul sub}
    \psplot[plotpoints=2000,linewidth=1.25pt,linecolor=blue]{-2}{6.5}{x 3 sub dup mul 1 sub}
    \rput(-1.5,10){$C_2$}\rput(6.8,10){\blue $C_1$}
\end{pspicture*}}\hfill
\parbox{0.58\linewidth}{\begin{enumerate}
    \item[1.] Laquelle de ces deux représentations est celle de la fonction $f$ ?
    \item[2.] Que vaut $f(3)$ ?
    \item[3.] Calculer le nombre qui a pour image 6 par la fonction $f$.
\end{enumerate}
}

\medskip

\begin{tabularx}{\linewidth}{|c|*{7}{>{\centering \arraybackslash}X|}}\hline
	&A		&B		&C		&D	&E	&F	&G\\ \hline
1	&$x$	&$- 2$	&$- 1$	&0	&1	&2	&3\\ \hline
2	&$f(x)$	&		&		&	&	&	&\\ \hline
\end{tabularx}

\medskip
{\begin{enumerate}
    \item[4.] La feuille de calcul ci -dessous permet de calculer des images par la fonction $f$.\\
    Quelle formule peut-on saisir dans la cellule B2 avant de l'étirer vers la droite jusqu'à la cellule G2 ?
\end{enumerate}

\subsection*{Exercice 4 - 13 points}

\medskip

Pendant les vacances, Robin est allé visiter le phare Amédée.

\smallskip

\parbox{0.5\linewidth}{Lors d'une sieste sur la plage il a remarqué que le sommet d'un parasol était en parfait alignement avec le sommet du phare. Robin a donc pris quelques mesures et a décidé de faire un schéma de la situation dans le sable pour trouver une estimation de la hauteur du phare.

\bigskip

\begin{itemize}[label={$\bullet$}]
\item Les points B, J et R sont alignés.
\item (SB) et (BR) sont perpendiculaires.
\item (PJ) et (BR) sont perpendiculaires.
\end{itemize}
} \hfill \parbox{0.48\linewidth}{\psset{unit=1cm}
\begin{pspicture}(-1,0)(5,8)
%\psgrid
\psline(0.5,1)(0.6,7)(0.8,7)(0.9,1)
\psline(0.5,1)(4.5,1)
\psline[linestyle=dashed](4.5,1)(0.8,7)
\rput{90}(0.2,4){Phare}\rput(4.5,2){Parasol}\rput(4.7,1.5){Moi}
\psline{->}(4.7,1.5)(4.5,1)\psline{->}(4.5,1.8)(3.6,1.6)
\uput[u](0.7,7){S} \uput[dl](0.5,1){B} \uput[dl](3.6,1){J}
\uput[dr](4.5,1){R} \uput[ur](3.6,2.4){P}
\psline{<->}(0.5,0.4)(4.5,0.4)\uput[d](2.5,0.5){34,7 m}
\psline{<->}(3.2,1)(3.2,2.6)\uput[l](3.2,1.8){2,1 m}
\psline{<->}(3.5,0.8)(4.5,0.8)\uput[d](4,0.85){1,3 m} 
\pspolygon(3.5,2.6)(3.6,1.4)(3.4,1.4)
\psline(3.5,1.4)(3.5,1)
\psline(0.92,1.3)(1.2,1.3)(1.2,1)\psline(3.5,1.3)(3.8,1.3)(3.8,1)
\end{pspicture}}

\medskip

\begin{enumerate}
    \item[1.] Quelle hauteur, arrondie au mètre, va-t-il trouver à l'aide de son plan ? Justifier la réponse.
    \item[2.] Déterminer la longueur SR. 
    \item[3.] En déduire la longueur SP.
\end{enumerate}

\newpage

\subsection*{Exercice 5 - 14 points}


\medskip

\parbox{0.3\linewidth}{Voici un programme de calcul :}\hfill
\parbox{0.68\linewidth}{
\begin{tabular}{|l|}\hline
$\bullet~~$Choisir un nombre\\
$\bullet~~$Ajouter 1 à ce nombre\\
$\bullet~~$Calculer le carré du résultat\\
$\bullet~~$Soustraire le carré du nombre de départ au résultat précédent.\\
$\bullet~~$Écrire le résultat.\\ \hline
\end{tabular}
}

\medskip

\begin{enumerate}
\item On choisit 4 comme nombre de départ. Prouver par le calcul que le résultat obtenu avec le programme
est 9.
\item On note $x$ le nombre choisi.\newline
	Exprimer le résultat du programme en fonction de $x$.\\

    \item On admettra que le résultat du programme de calcul est égal à $2x +1$. \\
    Soit $f$ la fonction définie par $f(x) = 2x + 1$.
	\begin{enumerate}
		\item Calculer l'image de 0 par $f$.
		\item Déterminer par le calcul l'antécédent de $5$ par $f$.
		\item En annexe, tracer la droite représentative de la fonction $f$.
		\item Par lecture graphique, déterminer le résultat obtenu en choisissant $- 3$ comme nombre de départ dans le programme de calcul. Sur l'annexe, laisser les traits de construction apparents.
	\end{enumerate}	
\end{enumerate}


\subsection*{Exercice 6 - 19 points}

\medskip

Avec un logiciel de géométrie, on exécute le programme ci-dessous.

\medskip

\parbox{0.45\linewidth}{Programme de construction :}\hfill\parbox{0.53\linewidth}{\qquad \qquad Figure obtenue:}

\parbox{0.45\linewidth}{\begin{itemize}[label={$\bullet$}]
    \item Construire un carré ABCD ;
    \item Tracer le cercle de centre A et de rayon [AC] ;
    \item Placer le point E à l'intersection du cercle et de la demi-droite [AB) ;
    \item Construire un carré DEFG.
\end{itemize}}\hfill
\parbox{0.53\linewidth}{ 
\psset{unit=1cm}
\begin{pspicture}(-2.2,-2)(4.2,4.2)
%\psgrid
\psframe[fillstyle=solid,fillcolor=lightgray](1.414,1.414)
\rput{55.5}(2,0){\psframe[fillstyle=solid,fillcolor=lightgray](0,0)(2.4495,2.4495)}
\psframe[fillstyle=solid,fillcolor=lightgray](1.414,1.414)
\pscircle(0,0){2}
\pspolygon[fillstyle=solid,fillcolor=gray](0,1.414)(1.414,0.414)(1.414,1.414)
\psline(4.2,0)

\uput[dl](0,0){A} \uput[d](1.414,0){B} \uput[ur](1.414,1.414){C} 
\uput[ul](0,1.414){D} \uput[dr](2,0){E} \uput[r](3.4,2){F} 
\uput[u](1.414,3.4){G}
\rput(1.1,4){} 
\end{pspicture}
}

\begin{enumerate}
\item Sur la copie, réaliser la construction avec AB $=3$~cm.
\item Dans cette question, AB $= 10~$cm.
	\begin{enumerate}
		\item Montrer que AC $= \sqrt{200}$~cm.
		\item Expliquer pourquoi AE $= \sqrt{200}$~cm.
		\item Montrer que l'aire du carré DEFG est le triple de l'aire du carré ABCD.
	\end{enumerate}
\item On admet pour cette question que pour n'importe quelle longueur du côté [AB],
l'aire du carré DEFG est toujours le triple de l'aire du carré ABCD.
	
En exécutant ce programme de construction, on souhaite obtenir un carré DEFG ayant
une aire de 48 cm$^2$.
	
Quelle longueur AB faut-il choisir au départ ?
\end{enumerate}


\newpage

\subsection*{Exercice 7 - 16 points}

\medskip

Un site internet propose de télécharger légalement des clips vidéos. Pour cela, sur la page d'accueil, trois choix s'offrent à nous:

\setlength\parindent{8mm}
\begin{itemize}[label={$\bullet$}]
    \item Premier choix : téléchargement \textbf{direct sans inscription}. Avec ce mode, chaque clip peut être téléchargé pour 4~euros.
    \item Deuxième choix: téléchargement \textbf{membre}. Ce mode nécessite une inscription à 10~euros.
    valable un mois et permet d'acheter par la suite chaque clip pour 2 euros.
    \item Troisième choix : téléchargement \textbf{premium}. Une inscription à 50~euros permettant de télécharger tous les clips gratuitement pendant un mois.
\end{itemize}
\setlength\parindent{0mm}

\medskip

\begin{enumerate}
\item Je viens pour la première fois sur ce site et je souhaite télécharger un seul clip.

Quel est le choix le moins cher ?
\item Pour cette question, utiliser l'annexe.
	\begin{enumerate}
		\item Compléter le tableau.
		\item À partir de combien de clips devient-il intéressant de s'inscrire en tant que membre ?
	\end{enumerate}
\item Dans cette question, $x$ désigne le nombre de clips vidéos achetés.
	
$f,\: g$ et $h$ sont trois fonctions définies par :
	
\setlength\parindent{8mm}
\begin{itemize}[label={$\bullet$}]
    \item $f(x) = 50$
    \item $g(x) = 4x$
    \item $h(x) = 2x + 10$
\end{itemize}
\setlength\parindent{0mm}

	\begin{enumerate}
		\item Associer chacune de ces fonctions au choix qu'elle représente (direct, membre ou premium).
		\item Dans le repère de l'annexe 2, tracer les droites représentant les fonctions $f,\: g$ et $h$.
		\item À l'aide du graphique, déterminer le nombre de clips à partir duquel l'offre premium devient la moins chère.
	\end{enumerate}
\end{enumerate}

\vspace{0,5cm}

\newpage \null 

\newpage

\textbf{Numéro de Candidat : } \fbox{\phantom{$\dfrac{10000000000000000000000000000}{0}$}}

\textbf{Annexe - Exercice 5}

\begin{center}

\psset{unit=0.5cm}
\begin{pspicture}(-10,-9)(10,9)
    \psgrid[gridlabels=0pt,subgriddiv=1,griddots=10]
    \psaxes[linewidth=1.25pt,Dx=2,Dy=2]{->}(0,0)(-10,-9)(10,9)
    \uput[u](9.5,0){$x$}\uput[r](0,9.5){$y$}
\end{pspicture}

\end{center}

\textbf{Annexe - Exercice 8}

\begin{center}    
    
\begin{tabularx}{\linewidth}{|m{3cm}|*{5}{>{\centering \arraybackslash}X|}}\hline
    Nombre de clips &1 &2 &5 &10 &15\\ \hline
    Prix en euros pour le téléchargement direct&4 &8&&&\\ \hline
    Prix en euros pour le téléchargement membre&12 &14&&&\\ \hline
    Prix en euros pour le téléchargement premium&50 &50&&&\\ \hline
\end{tabularx}
    
    
\psset{xunit=0.5cm,yunit=0.1cm}
\begin{pspicture}(-1,-10)(24,65)
    \multido{\n=0+1}{25}{\psline[linewidth=0.2pt](\n,0)(\n,60)}
    \multido{\n=0+5}{13}{\psline[linewidth=0.2pt](0,\n)(24,\n)}
    \psaxes[linewidth=1.25pt,Dy=100]{->}(0,0)(24,60)
    \multido{\n=0+5}{13}{\uput[l](0,\n){\n}}
    \uput[d](20.5,-4){Nombre de clips achetés}
    \uput[r](0,62.5){Prix en euros}
\end{pspicture}

\end{center}

\end{document}
